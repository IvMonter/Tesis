\chapter{Propiedades topológicas de gráficas}
En este capítulo dada una gráfica $G$ le asignaremos un espacio topológico por medio del complejo simplicial de completas de $G$, dicho espacio topológico es conocido como la realización geométrica de $G$.
Posteriormente estudiaremos las propiedades topológicas de $G$, con especial énfasis en las propiedades que se deducen o implican propiedades combinatorias.

\section{Realización geométrica de una gráfica}
\begin{Defi}
Sea $G$ una gráfica, consideremos el complejo simplicial abstracto de completas $G$, denotado por $\Delta(G)$. Definimos la realización geométrica de $G$ como la realización geométrica de $\Delta(G)$ y la denotamos por $\abs{\Delta(G)}$
\end{Defi}
Cuando hablemos de propiedades topológicas de una gráfica nos estaremos refiriendo a las propiedades topológicas de $\abs{\Delta(G)}$. Por ejemplo diremos que que $G$ es contraíble si $\abs{\Delta(G)}$ lo es.

Como estamos considerando gráficas finitas, la topología en $\abs{\Delta(G)}$ es la misma que la topológia usual como subespacio de $\mathbb{R}^N$, para algún $N$. Por lo tanto $\abs{\Delta(G)}$ es compacto.

\section{Gráficas buenas}
Uno de los objetivos principales de este trabajo es estudiar condiciones sobre una gráfica $G$ que aseguren que $G$ es homotópica a $K(G)$.
\begin{Defi}
Una gráfica $G$ es buena si $G\simeq K(G)$ y es muy buena si $G\simeq K^{n}(G)$ para toda $n\geq 1 $.
\end{Defi}

\section{Gráficas desmantelables}
\begin{Defi}
Una gráfica $G$ es desmantelable si sus vértices pueden ordenarse $V(G) = \{x_1,\ldots,x_n\}]$ de manera que $x_i$ es dominado en $G[\{x_i,\ldots,x_n\}]$ para toda $i = 1,\ldots,n-1$.
\end{Defi}

\section{Convergencia de gráficas de clanes}
\begin{Defi}
Una gráfica $G$ es clan-convergente si la sucesión de ordenes de gráficas $\{o(K^{n}(G)): n = 0,1,2,\ldots,n\}$ es acotada. En caso contrario decimos que la gráfica es divergente.
\end{Defi}


% Local Variables:
% TeX-master: "tesis.tex"
% End:
