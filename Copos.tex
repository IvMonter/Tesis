\section{Conjuntos parcialmente ordenados}
\begin{Teo}
Sea $P$ un conjunto parcialmente ordenado, estableciendo un orden en $P\times\{0,1\}$ tal que $(p,b)\leq (p^{'},b^{'})$ si y sólo si $p\leq p^{'}$ y $b\leq b^{'}$ se tiene que $\abs{P\times\{0,1\}}$ es homeomorfo a $\abs{P}\times[0,1]$
\end{Teo} 

\begin{Dem}

Consideremos las proyeccciones
\begin{align*}
&\pi_1\colon P\times\{0,1\} \rightarrow P\\
&\pi_2\colon P\times\{0,1\} \rightarrow \{0,1\}
\end{align*}

Como $\pi_1$ y $\pi_2$ son mapeos simpliciales pueden ser extendidos a funciones continuas $\phi_1$ y $\phi_2$ respectivamente tales que:
\begin{align*}
&\phi_1\colon\abs{P\times\{0,1\}} \rightarrow \abs{P}\\
&\phi_2\colon\abs{P\times\{0,1\}} \rightarrow [0,1]
\end{align*}

Sea $f\colon\abs{P\times \{0,1\}} \rightarrow \abs{P}\times [0,1]$ dada por 
\begin{equation*}
f(z)= (\phi_1(z),\phi_2(z))
\end{equation*}
f es continua por ser $\phi_1$ y $\phi_2$ continuas, además $f$ es inyectiva ya que si $f(z)=f(\hat{z})$ podemos escribir a $z$ y $\hat{z}$ como:
\begin{eqnarray}
z = \sum_{i=0}^{n}t_i(p_i,b_i)\\
\hat{z} = \sum_{i=0}^{m}\hat{t_i}(\hat{p_i},\hat{b_i})
\end{eqnarray} 
con $t_i>0$ y $\hat{t_i}>0$ para toda $i$, $(p_i,b_i)<(p_j,b_j)$ y $(\hat{p_i},\hat{b_i})<(\hat{p_j},\hat{b_j})$ para toda $i<j$.

Luego $f(z) = f(\hat{z})$ si y solo si $\phi_1(z)=\phi_1(\hat{z})$ y $\phi_2(z)=\phi_2(\hat{z})$, donde 
\begin{eqnarray}
\phi_1(z) = \sum_{i=0}^{n}t_ip_i = \phi_1(\hat{z}) = \sum_{i=0}^{m}\hat{t_i}\hat{p_i}\\
\phi_2(z) = \sum_{i=0}^{n}t_ib_1 = \phi_2(\hat{z}) = \sum_{i=0}^{m}\hat{t_i}\hat{b_i}
\end{eqnarray}
Como $t_i>0$ y $\hat{t_i}>0$ por unicidad de las coordenas baricéntricas $m=n$ y $t_i = \hat{t_i}$ para toda $i$, por lo tanto $z = \hat{z}$.

Para verificar que $f$ es un homeomorfismo como $\abs{P\times\{0,1\}}$ es compacto y $\abs{P}\times[0,1]$ es Hausdorff basta probar que f es sobreyectiva. 

Sea $p$ en $\abs{P}$ y $b$ en [0,1], entonces podemos escribir a $p$ y $b$ como:
\begin{eqnarray}
p = \sum_{i=0}^{n}t_ip_i \\
b = s\cdot 0 + (1-s)\cdot 1 = 1-s, 
\end{eqnarray}
con $t_i>0$, $\sum_{i=0}^{n}t_i =1$, $p_i<p_{i+1}$ y con $s$ en $[0,1]$

Sea $\{t_0,...,t_k\} $ tal que $\sum_{i=0}^{k}t_i <s$ y $\sum_{i=0}^{k+1}t_i\leq s$.

Definamos

$z =t_0(p_0,0)+ ...+ t_k(p_k,0)+(s-\sum_{i=0}^{k}t_i)(p_{k+1},0)+(\sum_{i=0}^{k+1}t_i-s)(p_{k+1},1)+t_{k+2}(p_{k+2},1)+...+t_n(p_n,1)$ 

Como $t_0 +...+t_k+(s-\sum_{i=0}^{k})+(\sum_{i=0}^{k+1}-s)+t_{k+2}+...+t_n = \sum_{t=0}^{n}t_i = 1$, se tiene que $z$ pertenece a $\abs{P\times\{0,1\}}$.

Luego
\begin{eqnarray*}
\phi_1(z) = \sum_{i=0}^{k}t_ip_i + (s-\sum_{i=0}^{k})p_{k+1} +(\sum_{i=0}^{k+1}-s)p_{k+1} + \sum_{i = k+2}^{n}t_ip_i = \sum_{i=0}^{n}t_ip_i = p\\
\phi_2(z) = \sum_{i=0}^{k}t_i\cdot 0 + (s-\sum_{i=0}^{k})\cdot 0 +(\sum_{i=0}^{k+1}-s)\cdot 1 + \sum_{i = k+2}^{n}\cdot 1 = \sum_{i=0}^{n}t_i -1 = 1-s = b
\end{eqnarray*}

Por lo tanto $f(z)=(p,b)$ y f es sobreyectiva.
\end{Dem}

\begin{Teo}
Sean P y Q conjuntos finitos parcialmente ordenados, y sean $f$ y $g$ funciones de P en Q tales que $f(x)\leq g(x), \forall x\in P$. Entonces $\abs{f}$ y $\abs{g}$ son homotópicas.
\end{Teo}
\begin{Dem}

Definamos $H\colon P\times \{ 0,1\} \rightarrow Q$ como:

$H(x,i)= \left\{ \begin{array}{lcc}
              f(x)&   si &i=0,  \\
             \\ g(x) &  si& i=1. 
             \end{array}
    \right. $

Luego $(x,i) \leq (y,j)$ si y solo si $x\leq y$ y $i\leq j$

Si $i=j$ como $f(x)\leq f(y)$ y $g(x)\leq g(y)$ se tiene que $H(x,i)\leq H(y,j)$

Si $i<j$, entonces $i = 0$ y $j = 1$, por lo que
$H(x,i) = f(x)\leq f(y)\leq g(y) = H(y,j)$

Por lo tanto $H(x,i)$ es un mapeo de conjuntos parcialemte ordenados por lo que puede ser extendida a una función continua $\abs{H}:\abs{P\times \{ 0,1\}} \longrightarrow \abs{Q}$ tal que si
\[
(x,r) =\sum_{i=1}^{n}t_{i}(p_{i},b{i}) \Longrightarrow \abs{H}((x,r))= \sum_{i=1}^{n}t_{i}H((p_{i},b{i}))
\]

Con $\sum_{i=1}^{n}t_{i}=1$, $t_{i}>0$ y $p_{i}$ en $P$, para toda $i$.

Ahora notemos que para $(x,0)$ se tiene que  $b_{i}=0$ para toda $i$ y para $(x,1)$ se tiene que $ b_{i} = 1$ para toda $i$ por lo tanto
\begin{eqnarray}
\abs{H}((x,0)) =\sum_{i=1}^{n}t_{i}f(p_{i}) = \abs{f}(x),\\
\abs{H}((x,1)) =\sum_{i=1}^{n}t_{i}g(p_{i}) = \abs{g}(x)
\end{eqnarray}

Como $\abs{P\times\{0,1\}}$ es homeomorfo a $\abs{P}\times[0,1]$ existe una función continua, invertible y con inversa continua $h\colon\abs{P}\times[0,1]\rightarrow \abs{P\times\{0,1\}}$ tal que $h(x,0) = (x,0)$ y $h(x,1) = (x,1)$.

Definamos $\hat{\abs{H}}\colon\abs{P}\times[0,1]\to \abs{Q}$ como $\hat{\abs{H}} =  \abs{H} \circ h$, dado que $h$ y $\abs{H}$ son continuas $\hat{\abs{H}}$ también lo es.

Finalmente 
$\hat{\abs{H}}(x,0)= \abs{H}(h((x,0)))=\abs{H}((x,0)) = \abs{f}(x)$,
$\hat{\abs{H}}(x,1)= \abs{H}(h((x,1)))=\abs{H}((x,1)) = \abs{g}(x)$.
\end{Dem}

\begin{Col}
Sea P un conjunto parcialmente ordenado con un elemento máximo, entonces $\abs{P}$ es contraíble.
\end{Col}
\section{Otro teorema}
\begin{Teo}
Sea $X$ un espacio contraíble y $f$ una función de $S^n$ en $X$. Entonces f puede extenderse a una función de $D^{n+1}$ en X.
\end{Teo}
 
\begin{Dem}
Como $X$ es contraíble, existe $H\colon X\times I \rightarrow X$ tal que:
\begin{eqnarray}
H(x,0)=x \\
H(x,1) = x_0
\end{eqnarray}
para toda x en X.

Definamos $G\colon S^n\times I \rightarrow X$ como:
\begin{equation}
G(u,t) = H(f(u),t)
\end{equation}
En $S^n\times I$ definimos una relación por $(u,1)\sim(u^{'},1)$ para toda $u$, $u^{'}$ en $S^n$
Ahora veamos que los elementos de la clase de equivalencia de $(u,0)$ bajo $G$ son enviados al mismo elemento en $X$.

Si $u$, $u^{'}$ pertenecen a $S^n$ se tiene que
\begin{eqnarray}
G(u,1) = H(f(u),1) = x_0\\
G(u^{'},1)=H(f(u^{'},1) = x_0
\end{eqnarray}
Por lo tanto existe $\bar{G}$ continua de $S^n\times I /{\sim}$ en $X$.
\end{Dem}