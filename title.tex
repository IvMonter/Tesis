\mainmatter 
\begin{titlepage}
  \begin{center}
    \null
    \vspace*{\fill}

    \includegraphics[scale=1.2,bb=55 20 0 0]{escudouaeh.pdf}

    \vspace*{\elespacio}

    \textsc{Universidad Autónoma del Estado de Hidalgo}

    \textsc{Instituto de Ciencias Básicas e Ingeniería}

    \textsc{Área Académica de Matemáticas y Física}

    \vspace*{\elespacio}

    {\Huge\bfseries Título\par}

    \vspace*{\elespacio}

    {\large Tesis que para obtener el título de}

    \vspace*{\elespacio}

    {\Large\textsc{Licenciada en Matemáticas Aplicadas}}

    \vspace*{\elespacio}

    {\large presenta}

    \vspace*{\elespacio}

    {\Huge Ivonne Monter Aldana}

    \vspace*{\elespacio}

    {\large bajo la dirección de}

    \bigskip

    {\Large Dr.~Rafael Villarroel Flores}

    \bigskip

    {Pachuca, Hidalgo. 2019}

    \vspace*{\fill}
  \end{center}
\end{titlepage}

\thispagestyle{empty}
\begin{flushleft}
  {\bfseries\Large Resumen}
\end{flushleft}

A cada gráfica simple y finita se le puede asociar un espacio topológico, por un procedimiento llamado realización geométrica. En esta tesis se estudia el efecto del operador de clanes en el tipo de homotopía de dicho espacio. Tenemos particular interés en las propiedades topológicas que se deducen de o implican propiedades combinatorias.

\vspace{2cm}

\begin{flushleft}
  {\bfseries\Large Abstract}
\end{flushleft}

A topological space can be associated to each simple and finite graph, by a procedure called geometric realization. In this thesis the effect of the clans operator on the type of homotopy of this space is studied. We have particular interest in the topological properties that are deduced from or imply combinatorial properties.



 \newpage \thispagestyle{empty}