%%	 -----------------------------------------------------------------------------------------------
%%	LIBRERIAS EMPLEADAS EN EL DOCUMENTO
%%	 -----------------------------------------------------------------------------------------------
\usepackage[utf8]{inputenc}
\usepackage[spanish]{babel}
\usepackage{amsmath}
\usepackage{amsfonts}
\usepackage{amssymb}
\usepackage{amsthm}
\usepackage{hyperref}
\usepackage{enumerate}
\usepackage{tikz}
\usepackage{caption}
\usepackage{cite}
\usepackage{amssymb,latexsym}
\usepackage{graphicx}
\usepackage{color}
\usepackage{tikz}
\usepackage{tkz-berge}
\usepackage{makeidx}
\usepackage{url}
\usepackage{xspace}
\usepackage{tocbibind}
\usepackage{hyperref}
% ver http://gilmation.com/articles/latex-margins-for-book-binding/
% y http://tex.stackexchange.com/questions/50258/margins-of-book-class
\usepackage[margin=3.5cm]{geometry}
\geometry{bindingoffset=1cm}



%%	 -----------------------------------------------------------------------------------------------
%%	ENTORNOS DEFINICIÓN - TEOREMA - PROPOSICIÓN - COROLARIO - LEMA - OBSERVACIÓN - DEMOSTRACIÓN
%%	 -----------------------------------------------------------------------------------------------
\theoremstyle{definition}
\newtheorem{Defi}{Definición}[section]
\newtheorem{Teo}{Teorema}[section]
\newtheorem{Ejem}{Ejemplo}[section]
\newtheorem{Prop}{Proposición}[section]
\newtheorem{Col}{Corolario}[section]
\newenvironment{Dem}
    {
    Demostración.
    }
    {
    \begin{flushright}
    $\square$
    \end{flushright}
    }
%%	 -----------------------------------------------------------------------------------------------
%%	COMANDOS
%%	 -----------------------------------------------------------------------------------------------
\newcommand{\abs}[1]{\left| #1 \right|}

	

