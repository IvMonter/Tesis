\chapter{Preliminares}                % Print a "chapter" heading
\pagenumbering{arabic}                  % Start text with arabic 1
En este capítulo introduciremos los conceptos básicos y resultados importantes en teoría de gráficas y topología que son la base del presente trabajo.
\section{Gráficas}
\begin{Defi}[Gráfica]
Una gráfica $G$ es un par de conjuntos finitos 
\begin{eqnarray*}
G=(V(G),E(G)),\
\end{eqnarray*}
donde $E(G)\subseteq \{\{x,y\}\subseteq V(G):x \neq y\}$. El conjunto $V(G)$ es el conjunto de vértices de $G$ y $E(G)$ es el conjunto de aristas.
\end{Defi}
\begin{Defi}[Vértices adyacentes]
Sea $G$ una gráfica, si $e =\{x,y\}$ pertenece a $E(G)$ decimos que $x$ y $y$ son adyacentes y escribimos $x \sim y$.
\end{Defi}
 Notemos que de la definición se deduce que $x\nsim x$ y si $x\sim y$ entonces $y\sim x$, además para cualesquiera puntos $x,y$ en $V(G)$ existe a lo más una arista entre ellos. Este tipo de gráficas se conocen como gráficas simples, las gráficas que consideraremos serán de este tipo.  
Existen otras variaciones de gráficas en las que se permiten multiples aristas y lazos (aristas de un vértice en sí mismo) para ver más consulte \cite{Harary1969} 

\begin{Defi}
El orden de una gráfica $G$ es su número de vértices y lo denotamos por $\abs{G}$.
\end{Defi}
\begin{Defi}
Dado un vértice $x$ de un  gráfica $G$, su vecindad abierta se define como $N_{G}(x) = \{y\in V(G): y\sim x\}$ 
\end{Defi}
\begin{Defi}
Dado un vértice $x$ de un  gráfica $G$, decimos que $x$ es un vértice dominado si existe $y$ en $N_{G}(x)$ tal que $N_{G}(x)$ está contenido en $N_{G}(y)\cup\{y\}$.
\end{Defi}

\begin{Defi}[Isomorfismo]
Sean $G$ y $H$ dos gráficas, $G$ y $H$ son isomorfas si existe una función biyectiva de los vértices de una en la otra tal que preserva las adyacencias. En tal caso escribimos $G\cong H$.
\end{Defi}
En principio los elementos del conjunto de vértices de una gráfica son arbitrarios, pero dado que las propiedades generales de gráficas son invariantes bajo isomorfismo podemos representar geométricamente una gráfica sin perder generalidad. Lo usual es pensar en los vértices como puntos en $\mathbb{R}^2$ y las aristas como lineas rectas entre ellos.

\begin{Defi}[Completa]
Sea $G$ una gráfica, una completa de $G$ es un subconjunto $c$ de $V(G)$ tal que cualesquiera dos elementos de $c$ son adyacentes.
\end{Defi}

\begin{Defi}[Clan]
Sea $G$ una gráfica, un clan $q$ de $G$ es una completa maximal bajo inclusión.
\end{Defi}

\begin{Defi}[Gráfica de clanes]
Sea $G$ una gráfica, la gráfica de clanes $K(G)$ de G está dada por:
\begin{align*}
V(K(G))&=\{q: q \text{ es un clan de } G\} \\
E(K(G))&=\{\{q_1,q_2\}: q_1\neq q_2 \text{ y } q_1\cap q_2 \neq \varnothing \}
\end{align*}
\end{Defi}

\section{Espacios topológicos}
%\begin{Defi}
%Una topología sobre un conjunto $X$ es una colección de conjuntos 
%\end{Defi}
\begin{Defi}
Dos espacios topológicos son equivalentes homotópicamente si existen mapas continuos  $f\colon X\rightarrow Y$,$g\colon Y\rightarrow X$ (llamados equivalencias homotópicas) tal que $f\circ g$ es homotópica a la identidad en $Y$ y $g\circ f$ es homotópica a la identidad en $X$
\end{Defi}

\begin{Defi}
Un espacio $X$ es contraíble si es homotópicamente equivalente a un punto.
\end{Defi}
\begin{Teo}
Sea $X$ un espacio contraíble y $f$ una función de $S^n$ en $X$. Entonces f puede extenderse a una función de $D^{n+1}$ en X.
\end{Teo}
 
\begin{Dem}
Como $X$ es contraíble, existe $H\colon X\times I \rightarrow X$ tal que:
\begin{eqnarray}
H(x,0)=x \\
H(x,1) = x_0
\end{eqnarray}
para toda x en X.

Definamos $G\colon S^n\times I \rightarrow X$ como:
\begin{equation}
G(u,t) = H(f(u),t)
\end{equation}
En $S^n\times I$ definimos una relación por $(u,1)\sim(u^{'},1)$ para toda $u$, $u^{'}$ en $S^n$
Ahora veamos que los elementos de la clase de equivalencia de $(u,0)$ bajo $G$ son enviados al mismo elemento en $X$.

Si $u$, $u^{'}$ pertenecen a $S^n$ se tiene que
\begin{eqnarray}
G(u,1) = H(f(u),1) = x_0\\
G(u^{'},1)=H(f(u^{'},1) = x_0
\end{eqnarray}
Por lo tanto existe $\bar{G}$ continua de $S^n\times I /{\sim}$ en $X$.
\end{Dem}